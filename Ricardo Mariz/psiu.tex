\documentclass[11pt,	 papera4]{article}
\usepackage[utf8]{inputenc}
\usepackage{amsmath}
\usepackage{amsfonts}
\usepackage{amssymb}
\usepackage{booktabs}
\usepackage[T1]{fontenc}
\usepackage{graphicx}
\usepackage{longtable}
\usepackage{float}
\usepackage{wrapfig}
\usepackage{soul}
\usepackage{amssymb}
\usepackage{hyperref}


\begin{document}


\title{PSIU - Protocolo Simples de Intercomunicação
Unificado}
\maketitle

\part*{Introdução \newline}

\paragraph*{} Um protocolo de comunicação nada mais é do que um conjunto de convenções que rege o tratamento e, especialmente, a formatação dos dados num sistema de comunicação. Seria a "gramática" de uma "linguagem" de comunicação padronizada. Conhecemos vários protocolos de comunicação e fazemos uso deles diariamente, mas não pensamos neles como protocolos de comunicação. O mais antigo deles é a língua falada: duas pessoas que emitem sons audíveis aos ouvidos humanos podem se comunicar. Neste exemplo, o protocolo de comunicação é a emissão de sons numa dada faixa de frequência, o código utilizado é a língua falada e a mensagem é o conteúdo do que se fala. Conforme o site Informática na Aldeia [1], \\

Na robótica recorremos o tempo todo a protocolos de comunicação para enviar comandos à robôs, obter informação de sensores, efetuar telemetria de dados, etc. De uma forma não unificada, na maioria dos casos, cada projetista cria seus próprios comandos e desenvolvem programas para controlar seus robôs a partir de computadores, tabletes ou outros dispositivos ou mesmos para robôs trocar informações entre si. 
Como esses comandos não são normatizados, eles acabam ficando limitados ao controle de poucos robôs desenvolvidos pela mesma pessoa ou empresa e não permite que os robôs se comuniquem entre si.\\
\\ \\

A proposta do PSIU é normatizar comandos, informações e formas de comunicação para que a conversa ocorra entre dispositivos independente dos programas de controle utilizados.\\ \\ \\

Entre inúmeros protocolos de comunicação que já foram criados, o modelo OSI é universalmente adotado e utilizado inclusive no TCP/IP (protocolo da Internet). De forma a ilustrar a abrangência do PSIU, ele representaria uma única camada do modelo OSI, a de aplicação. Isso quer dizer que o protocolo PSIU não normaliza as camadas mais baixas que definem o canal de comunicação e a forma que a mensagem chega até o destinatário. Ele apenas define palavras formadas em um padrão já consolidado, chamado ASCII, e essas palavras podem ser traduzidas em ações ou informações para robôs independente de sua arquitetura.\\

Sendo assim o protocolo PSIU pode ser utilizado em um canal serial (UART232, comN) consolidado no Sistema operacional de um PC, por exemplo, a partir de um driver específico (FTDI) e com comunicação por cabo. Como pode utilizar um driver similar consolidado através de uma comunicação bluetooth. Os mesmos comandos também podem trafegar em textos dentro de mensagens HTML ou sobre qualquer outro protocolo. O que o PSIU vem normatizar é o padrão em que o texto dos comandos são formatados.\\

Para garantir o envio de comandos em dispositivos com processamento limitado também existe a versão simplificada do PSIU que descreve comandos a partir de códigos hexadecimais de 8 bits sem a utilização da codificação ASCII. \\ \\

\textsl{[1] Informática na Aldeia, http://www.numaboa.com.br/informatica/internet/526-protocolos, acessado em fevereiro 2012}



\newpage




\part*{PSIU ASCII\newline}

\paragraph{} O PSIU procura definir uma forma de comunicação para diferentes tipos
de aplicação e volume de dados e diferentes topologias. Todas os frames (mensagens de comandos ou de respostas) enviadas no PSIU ASCII são escritos em caractere ASCII. Um pacote básico de
informação neste protocolo é mostrado abaixo: \\


\begin{table}[ht]
	\centering
	\begin{tabular}{lllll}
		\toprule
		Destinatário & Tamanho & Mensagem & Remetente &CheckSum \\
		\bottomrule
	\end{tabular}
	\label{tab:formatoslatex} %label da tabela
\end{table}



\paragraph{}O campo destinatário é o nome de identificação contido em cada robô. O próprio robô conhece seu nome e portanto sabe se o destinatário está correto ou não.\\ O tamanho é o tamanho do pacote, ou seja, a quantidade total de bytes contido na mensagem. \\ 
O  campo mensagem será ou um comando ou uma resposta. Se o comando/resposta tiver parâmetros eles devem estar na mensagem, caso contrário, será mandada uma resposta de erro.  \\
CheckSum é a soma de todos os bytes da mensagem.

\paragraph*{}Um exemplo de uma mensagem de comando no PSIU está descrita abaixo:\\

\begin{table}[ht]
	\centering
	
	\begin{tabular}{l}
		
		 \textbf{MNERIM031parafrente 100 PC02024}\\
		
	\end{tabular}
	\label{tab:formatoslatex} %label da tabela
\end{table}

\textsl
{Onde: 
Destinatário = MNERIM , 
Tamanho = 029 , 
Comando = parafrente,   
Parâmetro = 100 , Remetente = PC,
CheckSum = 02031} \\ \\

\newpage
\part*{Campos do pacote} 

\textbf{Destinatário}
\paragraph{}Destinatário é o campo onde vai conter o nome do dispositivo a ser enviado a mensagem.
Este nome pode conter letra maísculas, minúsculas, numerés, espaços e caracteres especiais, desde que sejam caracteres ASCII. \\
Cada dispositivo sabe seu nome e com isso ele só irá ler as mensagens que sejam mandadas diretamente para eles. \\
A quantidade de bytes é definido pelo tamanho do nome do dispositivo. \\ 
O número de caracteres do nome também pode variar entre os dispositivos, mas como cada um sabe seu próprio nome é possível indicar onde começa o próximo campo (tamanho da mensagem).



\paragraph{\textbf{Tamanho} \newline \newline} 

Tamanho é um campo de 3 bytes fixo que vai conter o tamanho total de bytes da mensagem. Sendo assim, o tamanho varia de 0 até 999.
Se uma mensagem contém apenas dezenas de bytes, o tamanho de 3 bytes se mantem sempre fixo e no campo da centena será colocado um 0.\\

\textsl{Exemplo: 033 (Mensagem de 33 bytes)} \\

\paragraph{\textbf{Mensagem} \newline \newline} 
O campo mensagem pode ser um comando ou uma resposta, incluido com um novo campo chamado parâmetro, que existirá se o comando ou resposta conter parâmetros.

\paragraph{\textbf{Comando} \newline \newline} 
O campo comando vai conter o nome do comando a ser executado pelo dispositivo. Propriamente dito, é uma palavra chave que deve indicar intuitivamente a ação a ser tomada pelo destinatário. \\
No final de cada string de nome terá que existir um espaço " " (0x20). De forma ao despositivo determinar o final de um comando. \\

\paragraph{\textbf{Resposta} \newline \newline} 
O campo resposta é uma mensagem de retorno definida a um comando executado. \\
No final de cada string de resposta terá que existir um espaço " " (0x20). De forma ao despositivo determinar o final de uma resposta. \\

\paragraph{\textbf{Parâmetro} \newline \newline } 
Parâmetro não é um campo obrigatório. Ele depende se o comando contém parâmetros ou não.
Caso um comando ou resposta não possua parâmetro este campo deve ser deixado em branco.
Um parâmetro assim como o comando, deve conter no final um espaço em branco para determinar o final do parâmetro. 
Se o comando tiver dois ou mais parâmetros a separação entre eles é dada por espaço " " (0x20). Caracteres do tipo float têm a parte inteira separada da fracionária por ponto (1.25) 

\paragraph{\textbf{CheckSum} \newline \newline } 
CheckSum é um campo de 5 bytes fixos de tamanho e é a soma em uma variável inteira de 16 bits (desprezando o transporte "vai um") de todos os bytes da mensagem (ele não incluido). Essa soma é representada em decimal. \\ Como os 5 bytes são fixos, se a soma dos bytes der um número de 4 bytes, deverá ser acrescentado um 0 na frente, assim formando um CheckSum de 5 bytes.\\

\textsl{ Exemplo: \\\\ int checksum = 0; \newline char mensagem[] = "MNERIM029parafrente 100 ";\newline for (x = 0; $x
< strlen(mensagem)$
; x++) \\
 \hspace*{1cm}checksum = checksum + mensagem[x]; \newline printf("\%s\%d",  mensagem, checksum); 
} \\

\newpage



\part*{Respostas \newline}


Para um controle mais eficaz nas respostas dos comandos no protocolo PSIU ASCII, antes de cada resposta (sucesso, falha, distância calculada por um sensor, etc) será mandado também o comando que foi executado. \\ \\

\begin{table}[ht]
	\centering
	\begin{tabular}{llllll}
		\toprule
		Dest. & Tam. & ComandoExecutado & Resposta & Remetente &CheckSum \\
		\bottomrule
	\end{tabular}
	\label{tab:formatoslatex} %label da tabela
\end{table}

Isso facilita o entendimento das respostas pelos dispositivos. \\\\
Exemplo:\\\\ \hspace*{1cm} 
\textsc{Envio do Comando "parafrente":} \\\\ \hspace*{2cm} \texttt{MNERIM029parafrente 100 PC02031} 



\paragraph*{}
\hspace*{0.8cm}\textsc{Resposta (Sucesso):} \\\\ \hspace*{2cm}\texttt{PC035parafrente sucesso MNERIM02656} \hspace*{2cm} \\ \\ \\ \\


\newpage
\part*{Respostas PSIU ASCII \newline \newline}

\paragraph{\textbf{sucesso} \newline \newline}

Uma mensagem de sucesso é enviada se um comando válido foi executado. 
\newline
Exemplo:\\\\ \hspace*{0.5cm} 
\textsc{Envio da resposta:} \\\\ \hspace*{2cm} \texttt{PCXXX"nomedocomando" sucesso MNERIMXXXXX} \\

\paragraph{\textbf{falha} \newline \newline}

Uma mensagem de falha é enviada se a execução se um comando falhou.
Exemplo:\\\\ \hspace*{0.5cm} 
\textsc{Envio da resposta:} \\\\ \hspace*{2cm} \texttt{PCXXX"nomedocomando" falha MNERIMXXXXX} \\

\newpage

\part*{Comandos \newline}


\subsection*{parafrente \\\\}
O comando \textit{parafrente} possui um parâmetro inteiro que varia de 0 a 999.  
Este comando diz ao robô a quantidade em \textit{cm} que ele irá andar para frente.\\
Se o comando for executado corretamente será recebido uma resposta de OK, caso contrário será recebido uma resposta de FALHA. \\ 
\newline
 Exemplo:\\\\ \hspace*{0.5cm} 
\textsc{Envio do Comando:} \\\\ \hspace*{2cm} \texttt{MNERIM031parafrente 100 PC02024} \\


\begin{table}[h]
	\centering
	\begin{tabular}{p{2cm}p{1cm}p{2cm}p{2cm}p{1cm}p{2cm}}
		\toprule
		MNERIM & 031 & parafrente & 100 & PC & 02024 \\
		\midrule	
		Destinatário & Tam. & Comando & Parâmetro & Rem. & CheckSum \\
		\bottomrule
	\end{tabular}
	\label{tab:formatoslatex} %label da tabela
\end{table}

\paragraph*{\newline\newline}
\hspace*{0.8cm}\textsc{Resposta:} \\\\ \hspace*{2cm}\texttt{PC035parafrente sucesso MNERIM02656} \hspace*{2cm}\textbf{(SUCESSO)}

\begin{table}[h]
	\centering
	\begin{tabular}{p{1cm}p{1cm}p{3.2cm}p{2cm}p{2cm}}
		\toprule
		PC & 035 &parafrente sucesso  & MNERIM & 02656 \\
		\midrule	
		Dest. & Tam. & Resposta & Rem. & CheckSum \\
		\bottomrule
	\end{tabular}
	\label{tab:formatoslatex} %label da tabela
\end{table}

\hspace*{1.2cm} \texttt{PC033parafrente falha MNERIM02389} \hspace*{2.5cm}\textbf{(FALHA)}

\begin{table}[h]
	\centering
	\begin{tabular}{p{1cm}p{1cm}p{3cm}p{2cm}p{2cm}}
		\toprule
		PC & 033 & parafrente falha  & MNERIM & 02389 \\
		\midrule	
		Dest. & Tam. & Resposta & Rem. & CheckSum \\
		\bottomrule
	\end{tabular}
	\label{tab:formatoslatex} %label da tabela
\end{table}

\newpage

\subsection*{paratras \\\\}
O comando \textit{paratras} possui um parâmetro inteiro que varia de 0 a 999. 
Este comando diz ao robô a quantidade em \textit{cm} que ele irá andar para trás.\\
Se o comando for executado corretamente será recebido uma resposta de OK, caso contrário será recebido uma resposta de FALHA. \\
\newline
Exemplo: \\\\ \hspace*{0.5cm} 
\textsc{Envio do Comando:} \\\\ \hspace*{2cm}\texttt{MNERIM029paratras 100 PC01829} \\

\begin{table}[h]
	\centering
	\begin{tabular}{p{2cm}p{1cm}p{2cm}p{2cm}p{1cm}p{2cm}}
		\toprule
		MNERIM & 029 & paratras & 100 & PC & 01829 \\
		\midrule	
		Destinatário & Tam. & Comando & Parâmetro & Rem.& CheckSum \\
		\bottomrule
	\end{tabular}
	\label{tab:formatoslatex} %label da tabela
\end{table}

\paragraph*{\newline\newline}
\hspace*{0.8cm}\textsc{Resposta:} \\\\ \hspace*{2cm}\texttt{PC033paratras sucesso MNERIM02452
} \hspace*{2cm}\textbf{(SUCESSO)}

\begin{table}[h]
	\centering
	\begin{tabular}{p{1cm}p{1cm}p{3cm}p{2cm}p{2cm}}
		\toprule
		PC & 033 &paratras sucesso  & MNERIM & 02452 \\
		\midrule	
		Dest. & Tam. & Resposta & Rem. & CheckSum \\
		\bottomrule
	\end{tabular}
	\label{tab:formatoslatex} %label da tabela
\end{table}

\hspace*{1.2cm} \texttt{PC031paratras falha MNERIM02185} \hspace*{2.5cm}\textbf{(FALHA)}

\begin{table}[h]
	\centering
	\begin{tabular}{p{1cm}p{1cm}p{3cm}p{2cm}p{2cm}}
		\toprule
		PC & 031 &paratras falha  & MNERIM & 02185 \\
		\midrule	
		Dest. & Tam. & Resposta & Rem. & CheckSum \\
		\bottomrule
	\end{tabular}
	\label{tab:formatoslatex} %label da tabela
\end{table}

\newpage 

\subsection*{giradireita \\\\}
O comando \textit{giradireita} possui um parâmetro inteiro que varia de 0 a 360.  
Este comando diz ao robô a quantidade em \textit{graus} que ele irá girar para direita.\\
Se o comando for executado corretamente será recebido uma resposta de OK, caso contrário será recebido uma resposta de FALHA. \\
\newline
Exemplo: \\\\ \hspace*{0.5cm} 
\textsc{Envio do Comando:} \\\\ \hspace*{2cm} \texttt{MNERIM032giradireita 100 PC02118} \\

\begin{table}[h]
	\centering
	\begin{tabular}{p{2cm}p{1cm}p{2cm}p{2cm}p{1cm}p{2cm}}
		\toprule
		MNERIM & 032 & giradireita & 100 & PC & 02118 \\
		\midrule	
		Destinatário & Tam. & Comando & Parâmetro &Rem. & CheckSum \\
		\bottomrule
	\end{tabular}
	\label{tab:formatoslatex} %label da tabela
\end{table}

\paragraph*{\newline\newline}
\hspace*{0.8cm}\textsc{Resposta:} \\\\ \hspace*{2cm}\texttt{PC036giradireita sucesso MNERIM02750} \hspace*{2cm}\textbf{(SUCESSO)}

\begin{table}[h]
	\centering
	\begin{tabular}{p{1cm}p{1cm}p{3.3cm}p{2cm}p{2cm}}
		\toprule
		PC & 036 &giradireita sucesso  & MNERIM & 02750 \\
		\midrule	
		Dest. & Tam. & Resposta & Rem. & CheckSum \\
		\bottomrule
	\end{tabular}
	\label{tab:formatoslatex} %label da tabela
\end{table}

\hspace*{1.2cm} \texttt{PC034giradireita falha MNERIM02483} \hspace*{2.5cm}\textbf{(FALHA)}

\begin{table}[h]
	\centering
	\begin{tabular}{p{1cm}p{1cm}p{3cm}p{2cm}p{2cm}}
		\toprule
		PC & 034 &giradireita falha  & MNERIM & 02483 \\
		\midrule	
		Dest. & Tam. & Resposta & Rem. & CheckSum \\
		\bottomrule
	\end{tabular}
	\label{tab:formatoslatex} %label da tabela
\end{table}


\newpage

\subsection*{giraesquerda \\\\}
O comando \textit{giraesquerda} possui um parâmetro inteiro que varia de 0 a 360.
Este comando diz ao robô a quantidade em \textit{graus} que ele irá girar para esquerda.\\
Se o comando for executado corretamente será recebido uma resposta de OK, caso contrário será recebido uma resposta de FALHA. \\
\newline
Exemplo:\\\\ \hspace*{0.5cm} 
\textsc{Envio do Comando:} \\\\ \hspace*{2cm} \texttt{MNERIM033giraesquerda 100 PC02239} \\

\begin{table}[h]
	\centering
	\begin{tabular}{p{2cm}p{1cm}p{2cm}p{2cm}p{1cm}p{2cm}}
		\toprule
		MNERIM & 033 & giraesquerda & 100 & PC & 02239 \\
		\midrule	
		Destinatário & Tam. & Comando & Parâmetro & Rem. & CheckSum \\
		\bottomrule
	\end{tabular}
	\label{tab:formatoslatex} %label da tabela
\end{table}
\paragraph*{\newline\newline}
\hspace*{0.8cm}\textsc{Resposta:} \\\\ \hspace*{2cm}\texttt{PC037giraesquerda sucesso MNERIM02871} \hspace*{2cm}\textbf{(SUCESSO)}

\begin{table}[h]
	\centering
	\begin{tabular}{p{1cm}p{1cm}p{3.5cm}p{2cm}p{2cm}}
		\toprule
		PC & 037 &giraesquerda sucesso  & MNERIM & 02871 \\
		\midrule	
		Dest. & Tam. & Resposta & Rem. & CheckSum \\
		\bottomrule
	\end{tabular}
	\label{tab:formatoslatex} %label da tabela
\end{table}

\hspace*{1.2cm} \texttt{PC035giraesquerda falha MNERIM02604} \hspace*{2.5cm}\textbf{(FALHA)}

\begin{table}[h]
	\centering
	\begin{tabular}{p{1cm}p{1cm}p{3.5cm}p{2cm}p{2cm}}
		\toprule
		PC & 035 &giraesquerda falha  & MNERIM & 02604 \\
		\midrule	
		Dest. & Tam. & Resposta & Rem. & CheckSum \\
		\bottomrule
	\end{tabular}
	\label{tab:formatoslatex} %label da tabela
\end{table}

\newpage

\subsection*{quantoscomandos \\\\}
O comando \textit{quantoscomandos} pergunta ao dispositivo a quantidade de funções disponíveis. O retorno dessa função é um número com representando a quantidade de comandos do dispositivo.\\
\newline
Exemplo:\\\\ \hspace*{0.5cm} 
\textsc{Envio do Comando:} \\\\ \hspace*{2cm} \texttt{MNERIM032quantoscomandos PC02415} \\

\begin{table}[h]
	\centering
	\begin{tabular}{p{2cm}p{1cm}p{3cm}p{1cm}p{2cm}}
		\toprule
		MNERIM & 032 & quantoscomandos & PC & 02415 \\
		\midrule	
		Destinatário & Tam. & Comando & Rem. &CheckSum \\
		\bottomrule
	\end{tabular}
	\label{tab:formatoslatex} %label da tabela
\end{table}
\paragraph*{\newline\newline}
\hspace*{0.8cm}\textsc{Resposta:} \\\\ \hspace*{2cm}\texttt{PC032quantoscomandos 6 MNERIM02501
} 

\begin{table}[h]
	\centering
	\begin{tabular}{p{1cm}p{1cm}p{3cm}p{2cm}p{2cm}}
		\toprule
		PC & 032 & quantoscomandos 6  & MNERIM & 02501 \\
		\midrule	
		Dest. & Tam. & Resposta & Rem. & CheckSum \\
		\bottomrule
	\end{tabular}
	\label{tab:formatoslatex} %label da tabela
\end{table}

\newpage
\subsection*{exibecomandos \\\\}
O comando \textit{exibecomandos} pede ao dispositivo para exibir o comando desejado. 
Essa função tem como parâmetro um numero inteiro.\\ O retorno dessa função é o nome do comando.\\
\newline
Exemplo:\\\\ \hspace*{0.5cm} 
\textsc{Envio do Comando:} \\\\ \hspace*{2cm} \texttt{MNERIM031exibecomando 1 PC02126} \\

\begin{table}[h]
	\centering
	\begin{tabular}{p{2cm}p{1cm}p{3cm}p{1cm}p{2cm}}
		\toprule
		MNERIM & 031 & exibecomandos & PC & 02126 \\
		\midrule	
		Destinatário & Tam. & Comando & Rem. &CheckSum \\
		\bottomrule
	\end{tabular}
	\label{tab:formatoslatex} %label da tabela
\end{table}
\paragraph*{\newline\newline}
\hspace*{0.8cm}\textsc{Resposta:} \\\\ \hspace*{2cm}\texttt{PPC038exibecomando parafrente MNERIM03148} 

\begin{table}[h]
	\centering
	\begin{tabular}{p{1cm}p{1cm}p{3cm}p{2cm}p{2cm}}
		\toprule
		PC & 038 & exibecomando parafrente  & MNERIM & 03148 \\
		\midrule	
		Dest. & Tam. & Resposta & Rem. & CheckSum \\
		\bottomrule
	\end{tabular}
	\label{tab:formatoslatex} %label da tabela
\end{table}

\newpage

\part*{Comandos de Controle \newline}
	

\paragraph{\textbf{qualseunome} \newline \newline}
O comando \textit{qualseunome} pergunta ao dispositivo o seu nome e recebe como resposta o nome do dispositivo. \\
Se houver interesse em se comunicar com o dispositivo e
caso não se saiba o nome do dispositivo usa-se esse comando. \\
No campo \textit{DESTINATÁRIO} no lugar do nome irá ter duas interrogações "??" (0x3F3F).
\\

Exemplo:\\\\ \hspace*{0.5cm} 
\textsc{Envio do Comando:} \\\\ \hspace*{2cm} \texttt{??024qualseunome PC01654} \\

\begin{table}[h]
	\centering
	\begin{tabular}{p{2cm}p{1cm}p{2cm}p{1cm}p{2cm}}
		\toprule
		?? & 024 & qualseunome & PC & 01654 \\
		\midrule	
		Destinatário & Tam. & Comando & Rem. & CheckSum \\
		\bottomrule
	\end{tabular}
	\label{tab:formatoslatex} %label da tabela
\end{table}
\paragraph*{\newline\newline}
\hspace*{0.8cm}\textsc{Resposta:} \\\\ \hspace*{2cm}\texttt{PC028qualseunome MNERIM01988} 

\begin{table}[h]
	\centering
	\begin{tabular}{p{1cm}p{1cm}p{2cm}p{2cm}p{2cm}}
		\toprule
		PC & 028 & qualseunome  & MNERIM & 01988 \\
		\midrule	
		Dest. & Tam. & Resposta & Rem. & CheckSum \\
		\bottomrule
	\end{tabular}
	\label{tab:formatoslatex} %label da tabela
\end{table}





\newpage

\begin{table}[h]
	\centering
	\caption{Comandos do PSIU}
	\begin{tabular}{p{3cm}p{2.5cm}p{9cm}}
		\toprule
		Comando &  Parâmetro & Exemplo de comando \\
		\midrule	
		qualseunome & -- & 	??024qualseunome PC01654\\	
		quantoscomandos & -- & PC032quantoscomandos 7 MNERIM02502  \\
		exibecomandos & 1 int & PC038exibecomando parafrente MNERIM03148\\
		parafrente & 1 int (0 : 999)  & MNERIM031parafrente 100 PC02024 \\
		paratras & 1 int (0 : 999)  & MNERIM029paratras 100 PC01829 \\
		giradireita & 1 int (0 : 360)  & MNERIM032giradireita 100 PC02118 \\
		giraesquerda & 1 int (0 : 360)  & MNERIM033giraesquerda 100 PC02239 \\
		\bottomrule
	\end{tabular}
	\label{tab:formatoslatex} %label da tabela
\end{table}
	
\newpage

\part*{PSIU HEXA}


Um pacote básico de informação neste protocolo é mostrado abaixo, onde os campos remetente, destinatário, tamanho e CRC são de 1 byte cada, o que facilita a implementação deste protocolo por dispositivos pouco complexos.


\begin{center}
\begin{tabular}{lllll}
\hline
 Remetente  &  destinatário  &  tamanho  &  carga  &  CRC  \\
\hline
\end{tabular}
\end{center}



Os campos de remetente e destinatário dependem dos níveis mais baixos da comunicação e da topologia da rede: Uma comunicação ponto-a-ponto não requer nenhum dos 2 campos e portanto começaria a partir do tamanho. Uma rede com um mestre e vários escravos precisa apenas do endereço do escravo (seja como remetente ou destinatário). Numa rede ad-hoc ambos os campos são necessários.

O tamanho é o tamanho do pacote, ou seja: número de bytes de carga +4. Como o tamanho é um byte a carga pode ser de 0 até 251 bytes dependendo do tipo de pacote.

\section{Tipos de pacote}
\label{sec-1}



\begin{center}
\begin{tabular}{lrr}
\hline
 Tipo      &  tamanho (bytes)  &  carga (bytes)  \\
\hline
 ping      &                4  &              0  \\
 comando   &                5  &              1  \\
 nó        &                6  &              2  \\
 complexo  &          7 a 255  &        3 a 251  \\
\hline
\end{tabular}
\end{center}



Um pacote do tipo ping não tem carga nenhuma. Normalmente é usado para saber se o dispositivo está vivo: A manda um ping para B e B responde com um ping para A. Para os dispositivos mais burros, o tamanho é sempre 4, 5 ou 6. Apenas dispositivos com maior capacidade de processamento reconhecem um pacote completo.
\newpage
\subsection{Comando}
\label{sec-1.1}


Um pacote do tipo comando manda apenas 1 byte de informação. É usado para mensagens do tipo ACK, NAK, BUSY e AVAILABLE.

Uma possibilidade de valores de carga padrão para este tipo de pacote é:


\begin{center}
\begin{tabular}{lll}
 carga         &  comando  &  Significado                                                                                                                \\
\hline
 65 (ascii A)  &  ACK      &  Último pacote foi recebido corretamente.                                                                                   \\
 78 (ascii N)  &  NAK      &  Erro no último pacote (erro de CRC). Reenvie.                                                                              \\
 66 (ascii B)  &  BUSY     &  Não consegui processar o último pacote. Não reenvie pois estou ocupado.                                                    \\
 63 (ascii ?)  &  AVAIL?   &  Pergunta: pronto para receber um pacote?.                                                                                  \\
 70 (ascii F)  &  FREE     &  Resposta: pronto para receber um pacote. (Se não, manda um BUSY)                                                           \\
 87 (ascii W)  &  WTF      &  Último comando é impossível de ser executado (CRC não deu erro e não está ocupado, mas o valor de carga não faz sentido).  \\
\hline
\end{tabular}
\end{center}



\subsection{Nó}
\label{sec-1.2}


No pacote tipo nó, um byte de informação é enviado a um nó específico do dispositivo. Cada nó é um atuador, um sensor ou um elemento mais abstrato mas que é modelado como um atuador ou um sensor. O valor 0 no campo nó é reservado para a resposta de sensores.


\begin{center}
\begin{tabular}{llllll}
\hline
 Remetente  &  destinatário  &  tamanho  &  nó  &  sinal  &  CRC  \\
\hline
\end{tabular}
\end{center}



\subsubsection{Atuadores e sensores}
\label{sec-1.2.1}


Nós atuadores recebem um sinal e respondem com com um ACK.


\begin{center}
\begin{tabular}{ll}
\hline
 tipo do nó     &  sinal                                                       \\
\hline
 DC aberto      &  Valor de um PWM que controla o motor DC                     \\
 DC controlado  &  Velocidade do motor. 255 (ou 127) é a máxima vel. definida  \\
 stepper        &  Número de passos                                            \\
 servo          &  -128: -90o. 127: +90o                                       \\
 linear         &  valor entre 0 (mínimo) e 255 (máximo) de atuador linear     \\
 on-off         &  0 (desligado) ou 255 (ligado)                               \\
\hline
\end{tabular}
\end{center}



Quando o nó é um sensor, o campo sinal não importa. Na resposta o campo nó recebe valor 0 como indicador de que é uma resposta a um pedido de informação. Se um dispositivo recebe um pacote do tipo nó com nó igual a 0 sem ter pedido informação a nenhum outro dispositivo, ele deve responder com um WTF.


\begin{center}
\begin{tabular}{ll}
\hline
 tipo do nó         &  sinal  \\
\hline
 Luminosidade       &         \\
 ângulo             &         \\
 distância (sonar)  &         \\
 velocidade         &         \\
 aceleração         &         \\
 on-off             &         \\
\hline
\end{tabular}
\end{center}



\subsection{Complexo}
\label{sec-1.3}


Para dispositivos inteligentes ou muitos bytes de dados, um pacote pode ter vários bytes. Cada dispositivo sabe o tamanho do seu nome, logo vai considera o campo destinatário como sendo deste tamanho.

O tamanho máximo é 255 bytes. Se porventura houver uma mensagem maior, ela deverá ser quebrada em mensagens de 255 bytes.


\begin{center}
\begin{tabular}{lllllllll}
\hline
 rem.  &  dest.  &  tamanho  &  nCampos  &  tam. campo 1  &  \ldots{}  &  tam. campo n  &  payload  &  CRC  \\
\hline
\end{tabular}
\end{center}



Rementente é o robô que enviou a mensagem e destinatário é para quem é a mensagem. O campo de remetente é importante pois o destinatário não precisa obedecer a todo robô. Um byte permite endereçar 256 robôs.
Os nós são dispositivos de um robô: motores, sensores, etc. Em função dele o sinal pode significar diferentes coisas.

\subsubsection{Atuadores}
\label{sec-1.3.1}


\begin{center}
\begin{tabular}{ll}
\hline
 tipo do nó     &  sinal                                                       \\
\hline
 DC aberto      &  Valor de um PWM que controla o motor DC                     \\
 DC controlado  &  Velocidade do motor. 255 (ou 127) é a máxima vel. definida  \\
 stepper        &  Número de passos                                            \\
 servo          &  -128: -90o. 127: +90o                                       \\
 linear         &  valor entre 0 (mínimo) e 255 (máximo) de atuador linear     \\
 on-off         &  0 (desligado) ou 255 (ligado)                               \\
\hline
\end{tabular}
\end{center}


\newpage
\subsubsection{Sensores}
\label{sec-1.3.2}


Os sensores envolvem 2 sinais: um pedindo informação do sensor, com apenas os endereços e o nó, e o outro enviando esta informação, com o sinal relevante. Neste caso pode-se colocar um bit do nó como indicador se é um pedido ou uma resposta.


\begin{center}
\begin{tabular}{ll}
\hline
 tipo do nó         &  sinal  \\
\hline
 Luminosidade       &         \\
 ângulo             &         \\
 distância (sonar)  &         \\
 velocidade         &         \\
 aceleração         &         \\
 on-off             &         \\
\hline
\end{tabular}
\end{center}




\end{document}
